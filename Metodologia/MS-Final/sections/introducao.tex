\section{Introdução}
\label{sec:introducao}

No contexto histórico da computação, presenciamos uma incrível evolução tecnológica que revolucionou a forma como interagimos com as máquinas e processamos informações. De acordo com Hennessy e Patterson (2017), os primeiros computadores eletrônicos, como o ENIAC nos Estados Unidos e o Colossus no Reino Unido, foram desenvolvidos na década de 1940, sendo fundamentais para avanços científicos e militares.

Ao longo das décadas seguintes, ocorreram avanços significativos na tecnologia computacional e na miniaturização dos sistemas. Um exemplo disso foi a invenção do transistor em 1947 e seu subsequente desenvolvimento em circuitos integrados, impulsionando assim os esforços em direção ao potencial uso comercial dos sistemas.
A partir da década de 90, com o surgimento da World Wide Web, uma rede global de informações interconectadas, houve um impulso nas inovações relacionadas a processadores. A popularização da Web e o aumento do tráfego de dados online exigiram processadores mais rápidos e eficientes. As empresas de tecnologia investiram em pesquisa e desenvolvimento para melhorar o desempenho e a capacidade de processamento dos processadores.

No mercado atual, as arquiteturas x86 e ARM são amplamente adotadas, cada uma com suas características distintas. De acordo com Tanenbaum e Austin (2013), a arquitetura ARM é especialmente adequada para dispositivos de menor porte e com requisitos de dissipação de calor reduzidos, como smartphones e dispositivos de Internet das Coisas. Por outro lado, a arquitetura x86 é amplamente utilizada em várias aplicações de computação em geral, incluindo computadores pessoais e servidores.

Tanto a arquitetura x86 quanto a ARM se beneficiam dos avanços nas técnicas de fabricação de litografia. Esses processos permitem a criação de transistores em uma escala nanométrica, resultando em uma maior densidade e desempenho dos chips. A litografia é um método essencial para a produção de circuitos integrados, possibilitando a criação de componentes complexos em um tamanho reduzido.
A utilização de processos de fabricação de litografia na produção de chips tem impulsionado o avanço tecnológico, permitindo a criação de dispositivos eletrônicos cada vez mais poderosos e eficientes. A capacidade de fabricar transistores em uma escala tão pequena tem contribuído para o aumento da densidade de componentes e o aprimoramento do desempenho dos sistemas baseados em x86 e ARM.



No entanto, apesar dos avanços contínuos nas arquiteturas x86 e ARM e nas técnicas de fabricação de litografia, a indústria de semicondutores está enfrentando uma problemática relacionada ao limite físico dos componentes eletrônicos. Segundo Stallings (1987), à medida que os transistores e outros elementos dos circuitos integrados são miniaturizados, questões como vazamento de corrente, interferência eletromagnética e perda de sinal começam a se tornar cada vez mais relevantes.

Devido à escala nanométrica dos componentes, questões como vazamento de corrente, interferência eletromagnética e perda de sinal começam a se tornar cada vez mais relevantes. Além disso, o aumento na densidade de transistores em um chip resulta em um aumento na dissipação de calor, o que pode afetar negativamente o desempenho e a confiabilidade dos dispositivos.

Nesse contexto, torna-se fundamental buscar soluções viáveis e eficientes para superar esses obstáculos e impulsionar a evolução da tecnologia computacional.

O objetivo deste artigo é realizar um levantamento das possíveis soluções para os desafios enfrentados na construção de processadores. Serão analisadas diferentes abordagens e tecnologias, explorando seus pontos positivos e negativos. Compreender as alternativas disponíveis e suas implicações é essencial para direcionar o futuro desenvolvimento de processadores, possibilitando avanços tecnológicos que impulsionam a sociedade em direção a um novo patamar de capacidade de processamento e inovação.
